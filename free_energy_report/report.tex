\documentclass{article}
\usepackage[a4paper, total={6in, 8in}]{geometry}
\usepackage{amsmath}
\usepackage{hyperref}
\usepackage{graphicx}
\DeclareMathAlphabet\mathbfcal{OMS}{cmsy}{b}{n}

\title{Free energies from transition path sampling}
\date{\today}
\begin{document}
\maketitle
%
\section{Introduction}
Consider a molecular system evolving in time and snapshots of 
this system taken at regular intervals of $\Delta t$ for a total time of 
$\mathcal{T}$. The coordinates and momenta of all the $N$ atoms are defined as
$\textbf{q}=\{q_1,q_2,q_3,\ldots,q_N\}$ and $\textbf{p}=\{p_1,p_2,p_3,\ldots,p_N\}$, respectively. 
Complete description of the system at time step $i\Delta t$ is provided by the variable $\mathcal{Z}_{i\Delta t}=\{\mathbf{q}_{i\Delta t},\mathbf{p}_{i\Delta t}\}$ and for the entire time period this can be 
compactly represented by the time ordered set
\begin{equation}
\mathbfcal{Z} = \{\mathcal{Z}_0, \mathcal{Z}_{\Delta t}, \mathcal{Z}_{2\Delta t},
\ldots,\mathcal{Z}_{\mathcal{T}}\},
\end{equation}
which provides a complete description of a dynamical trajectory from time $t=0$ to $t=\mathcal{T}$.
The reactant and product states are conveniently defined using order parameter $\xi(\textbf{q})$ which 
depends on the atomic positions $\textbf{q}$. 

\section{Free energy}
The free energy $(A(\xi))$ as a function of a reaction coordinate $\xi$ is defined as 
\begin{equation}
A(\xi) = -k_{\text{B}}T\ln[P(\xi)],
\end{equation}
where $P(\xi)$ is the probability distribution of the reaction coordinate $\xi$ 
in the equilibrium ensemble with the distribution function $\rho(\textbf{q})$
\begin{equation}
P(\tilde{\xi}) = \int dV \rho(\textbf{q})\delta[\tilde{\xi}-\xi(\textbf{q})]
\end{equation}
Here $dV$ extends over the entire configuration space. In molecular dynamics simulations the 
distribution functions are approximated by discretized histograms that are calculated by determining 
how often the reaction coordinate $\xi({\textbf{q}})$ falls within the various histogram bins. 


%\section{Path sampling and free energies}

%\section{Preliminary results from WHAM}
%\begin{figure}
%\centering
%\includegraphics[scale=0.8]{figures/free_n_vs_r.pdf}
%\end{figure}

\section{Procedure to calculate free energies}
\begin{itemize}
\item Divide the order parameter variable into 6-10 non-overlapping windows
\item Run TPS simulations within each window
\begin{itemize}
	\item How to do this? Take a parent reactive trajectory that goes from the reactant to product. Define state A and B to be the boundaries of the window. Fix a smaller time within which the dynamics can take the system
	from A to B or B to A. Run TPS simulations accepting all trajectories? 
	\item 75 trajectories per window with 10 windows means 750 trajectories! Thats a large number and automation of the input and submission is necessary.
	Maybe using python or BASH
	\end{itemize}
\item The histograms of each window are obtained by running 75 trajectories per window
\item The WHAM procedure will be used to invert these histograms and obtain the free energy
	\end{itemize}


\bibliographystyle{unsrt}
\bibliography{report}
\end{document}
\documentclass{article}
\usepackage[a4paper, total={6in, 8in}]{geometry}
\usepackage{amsmath}

\title{Postdoc work preparation}
\date{\today}
\begin{document}
\maketitle
%
\section{Recent developments in methods for identifying reaction coordinates}
\begin{itemize}
\item Straightforward MD simulations sample stable regions
\item More interesting transition regions are rarely visited
\item Various enhanced sampling methods are used for studying these rare events such as 
umbrella sampling, metadynamics, orthogonal space sampling and constrained dynamics. For
improved sampling of regions other than stable basins.  
\item These methods rely on application of a biasing potential on one or a small set of 
coordinates, usually termed \emph{reaction coordinates} or \emph{order parameters},
along which the progress of the transition can be quantified to certain extent. 
\item Two different views: reaction coordinates should reveal the underlying mechanism of the process
under study and reaction coordinates should provide a reduced description of a given process that preserves
some geometric or informatic metric of the configuration space of the system. 
\item Free energy related definition and committor are the prominent examples of the first group. 
Committor is gaining popularity as the measure of the quality of reaction coordinates due to its clear
and specific relationship with reaction dynamics. 
\item Existing methods for finding the reaction coordinates are heavily trial and error in
nature: a structural coordinate is selected bason on chemical or phyiscal intuition,
then biased MD is performed along the proposed coordinate to collect necessary information
which is used to judge whether the proposed coordinate is reaction coordinate or not. 
\end{itemize}
Ok, this paper turned out to be a dud actually. Not very useful at all, just gave some 
really broad overview of the field. Time to trust the big guns in the field. 

\section{Transition path sampling}
This section is a summary of my reading of the book chapter Transition path sampling
written by Christoph Dellago, Peter Bolhuis and Phillip Geissler in \emph{Advances in chemical physics, 123:1-86}.
One example for importance of rare event sampling.
\begin{itemize}
\item A specific water molecule in liquid water has a lifetime of about 10 hours
before it dissociates to form hydronium and hydroxide ions.
\item In a system of 100 water molecules, there will only be a few dissociation events occuring every hour.
\item Simulation of molecular motions proceeds in time steps of roughly $1\;fs$, this means that 
$10^8$ timesteps would be required to observe just one such event. Such calculations are beyond capabilities of 
the fastest available computers today and in the foreseeable future. This is an example of 
simulating an extremely rare event. 
\end{itemize} 
\end{document}
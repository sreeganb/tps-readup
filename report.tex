\documentclass{article}
\usepackage[a4paper, total={6in, 8in}]{geometry}
\usepackage{amsmath}
\usepackage{hyperref}

\title{Postdoc work preparation}
\date{\today}
\begin{document}
\maketitle
%
\section{Recent developments in methods for identifying reaction coordinates}
\begin{itemize}
\item Straightforward MD simulations sample stable regions
\item More interesting transition regions are rarely visited
\item Various enhanced sampling methods are used for studying these rare events such as 
umbrella sampling, metadynamics, orthogonal space sampling and constrained dynamics. For
improved sampling of regions other than stable basins.  
\item These methods rely on application of a biasing potential on one or a small set of 
coordinates, usually termed \emph{reaction coordinates} or \emph{order parameters},
along which the progress of the transition can be quantified to certain extent. 
\item Two different views: reaction coordinates should reveal the underlying mechanism of the process
under study and reaction coordinates should provide a reduced description of a given process that preserves
some geometric or informatic metric of the configuration space of the system. 
\item Free energy related definition and committor are the prominent examples of the first group. 
Committor is gaining popularity as the measure of the quality of reaction coordinates due to its clear
and specific relationship with reaction dynamics. 
\item Existing methods for finding the reaction coordinates are heavily trial and error in
nature: a structural coordinate is selected bason on chemical or phyiscal intuition,
then biased MD is performed along the proposed coordinate to collect necessary information
which is used to judge whether the proposed coordinate is reaction coordinate or not. 
\end{itemize}
Ok, this paper turned out to be a dud actually. Not very useful at all, just gave some 
really broad overview of the field. Time to trust the big guns in the field. 

\section{Transition path sampling}
This section is a summary of my reading of the book chapter Transition path sampling
written by Christoph Dellago, Peter Bolhuis and Phillip Geissler in \emph{Advances in chemical physics, 123:1-86}.
One example for importance of rare event sampling.
\begin{itemize}
\item A specific water molecule in liquid water has a lifetime of about 10 hours
before it dissociates to form hydronium and hydroxide ions.
\item In a system of 100 water molecules, there will only be a few dissociation events occuring every hour.
\item Simulation of molecular motions proceeds in time steps of roughly $1\;fs$, this means that 
$10^8$ timesteps would be required to observe just one such event. Such calculations are beyond capabilities of 
the fastest available computers today and in the foreseeable future. This is an example of 
simulating an extremely rare event. 
\end{itemize} 

\begin{itemize}
\item Calulation of free energies using the transition path sampling method. This is supposed to be done with 
trajectories that were thrown away. 
\end{itemize}

\section{Learning CHARMM}
CHARMM is a molecular mechanics software developed at Harvard. Here I am going to use it to 
study enzyme catalysis mechanisms based on the transition path sampling method. Here is a summary of what
I have learnt so far about using charmm.

\begin{itemize}
	\item The pdb file cannot be read into CHARMM just like that. It needs some modifications. There is a script 
	called efixpdb.awk that used awk command line tool to edit the pdb file into a format that can be 
	read by CHARMM. 
	\item This script requires the identification and understanding of the system and subsystems at hand. 
	For example, the protein and the complex that its binding to, the solvent around the protein, and so on.
\end{itemize}

\section{BOLAS method for calculation of the free energy profile}

Transition path sampling does not provide a means for the sampling of equilibrium distribution of the 
free energy landscape. In order to sample the equilibrium distribution of trajectories, the ones that 
are thrown away needs to be sampled. 


\section{TPS}
A molecular dynamics trajectory of length $\mathcal{T}$ can be discretized 
for convenience as a time ordered sequence of states as
\begin{equation}
x(\mathcal{T}) = \{x_0,x_{\Delta t},x_{2\Delta t},\ldots x_{\mathcal{T}}\},
\end{equation}
with a time step of $\Delta t$ and $x_{i\Delta t}$ is a complete snaphot of the
system at time $i\Delta t$. For a system evolving under Newtonian dynamics, $x$
represents the positions and momentums of all the constituent particles. These 
discrete states are also known as time slices. 

 The probability to observe a particular trajectory $x(\mathcal{T})$ in phase space depends on the 
 initial conditions and the type of time evolution dynamics. In case of Markovian dynamics, the 
 probability to go from state $x_t$ to $x_{t+\Delta t}$ depends only on the 
 state $x_t$ and not on the entire history of the trajectory. For such dynamics, the 
 path probability can be written as a product of single time step transition probabilities
 $p(x_t\rightarrow x_{t+\Delta t})$,
 \begin{equation}
\mathcal{P}[x(\mathcal{T})] = \rho(x_0)\prod^{(\mathcal{T}/\Delta t)-1}_{i=0}p(x_i\rightarrow x_{(i+1)\Delta t}).
 \end{equation}
 $\rho(x_0)$ is the probability distribution of the initial conditions. The transition probability
 is different for deterministic and stochastic dynamics.  
\subsection{Reactive pathways}
Restrictions need to be placed for studying the reactive pathways. The path probability for 
reactive trajectories is defined as
\begin{equation}
\mathcal{P}_{AB}[x(\mathcal{T})]= h_A(x_0)\mathcal{P}[x(\mathcal{T})]h_B(x_{\mathcal{T}})/Z_{AB}(\mathcal{T})
\end{equation}
$h_{A/B}(x)$ are characteristic functions defined as
\[
 h_{A,B}(x) = 
  \begin{cases} 
   1 & \text{if } x \in A,B \\
   0 & \text{if } x \not\in A,B
  \end{cases}
\]
Since the path probability is restricted to reactive trajectories, the 
function needs to be renormalized for which $Z_{AB}(\mathcal{T})$ is used
\begin{equation}
Z_{AB}(\mathcal{T}) \equiv \int Dx(\mathcal{T})h_A(x_0)\mathcal{P}[x(\mathcal{T})]
h_B(x_{\mathcal{T}})
\end{equation}
where the notation $\int Dx(\mathcal{T})$ from the path integral theory is 
\begin{equation}
\int Dx(\mathcal{T}) \equiv \int\ldots\int dx_0dx_{\Delta t}dx_{2\Delta t}\ldots dx_{\mathcal{T}}
\end{equation}
The transition path ensemble is a complete statistical description of all possible pathways connecting reactants with products. 
\subsection{Initial conditions}
In a canonical ensemble the distribution of the initial conditions
at temperature $T$ is 
\begin{align}
\rho(x) &= \exp(-\beta\mathcal{H}(x))/Z \\
Z(\beta) &= \int dx\exp(-\beta\mathcal{H}(x))
\end{align} 
\subsection{Newtonian dynamics}
Hamilton equations of motion are given by
\begin{align}
\dot{r} &=\frac{\partial \mathcal{H}(r,p)}{\partial p} \\
\dot{p} &=-\frac{\partial\mathcal{H}(r,p)}{\partial r}
\end{align}
Since Newtonian dynamics is deterministic, the state of the system is entirely
determined by the initial state $x_0$ at time $t=0$. 
$x=\{r,p\}$, collectively represents the positions and momenta of the system. 
For such deterministic dynamics, the transition probability can be written in terms of the Dirac delta function

\begin{equation}
p(x_t\rightarrow x_{t+\Delta t}) = \delta(x_{t+\Delta t}-\phi_{\Delta t}(x_t))
\end{equation}

\subsection{Problems to address}

\begin{itemize}
\item Most of the trajectories are unreactive in a TPS simulation. How do we calculate the 
reversible work from all of these? Do we keep doing the simulation till we have converged 
free enrgies? 
\item TPS virtual replica exchange \cite{Brotzakis19JChemPhys151p174111} 
\item Recycling the rejected Monte-Carlo moves \cite{Frenkel17571}
\end{itemize}
\subsection{VIE-TPS}



\bibliographystyle{unsrt}
\bibliography{report}
\end{document}